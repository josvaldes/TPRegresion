% Options for packages loaded elsewhere
\PassOptionsToPackage{unicode}{hyperref}
\PassOptionsToPackage{hyphens}{url}
%
\documentclass[
]{article}
\usepackage{amsmath,amssymb}
\usepackage{lmodern}
\usepackage{iftex}
\ifPDFTeX
  \usepackage[T1]{fontenc}
  \usepackage[utf8]{inputenc}
  \usepackage{textcomp} % provide euro and other symbols
\else % if luatex or xetex
  \usepackage{unicode-math}
  \defaultfontfeatures{Scale=MatchLowercase}
  \defaultfontfeatures[\rmfamily]{Ligatures=TeX,Scale=1}
\fi
% Use upquote if available, for straight quotes in verbatim environments
\IfFileExists{upquote.sty}{\usepackage{upquote}}{}
\IfFileExists{microtype.sty}{% use microtype if available
  \usepackage[]{microtype}
  \UseMicrotypeSet[protrusion]{basicmath} % disable protrusion for tt fonts
}{}
\makeatletter
\@ifundefined{KOMAClassName}{% if non-KOMA class
  \IfFileExists{parskip.sty}{%
    \usepackage{parskip}
  }{% else
    \setlength{\parindent}{0pt}
    \setlength{\parskip}{6pt plus 2pt minus 1pt}}
}{% if KOMA class
  \KOMAoptions{parskip=half}}
\makeatother
\usepackage{xcolor}
\usepackage[margin=1in]{geometry}
\usepackage{color}
\usepackage{fancyvrb}
\newcommand{\VerbBar}{|}
\newcommand{\VERB}{\Verb[commandchars=\\\{\}]}
\DefineVerbatimEnvironment{Highlighting}{Verbatim}{commandchars=\\\{\}}
% Add ',fontsize=\small' for more characters per line
\usepackage{framed}
\definecolor{shadecolor}{RGB}{248,248,248}
\newenvironment{Shaded}{\begin{snugshade}}{\end{snugshade}}
\newcommand{\AlertTok}[1]{\textcolor[rgb]{0.94,0.16,0.16}{#1}}
\newcommand{\AnnotationTok}[1]{\textcolor[rgb]{0.56,0.35,0.01}{\textbf{\textit{#1}}}}
\newcommand{\AttributeTok}[1]{\textcolor[rgb]{0.77,0.63,0.00}{#1}}
\newcommand{\BaseNTok}[1]{\textcolor[rgb]{0.00,0.00,0.81}{#1}}
\newcommand{\BuiltInTok}[1]{#1}
\newcommand{\CharTok}[1]{\textcolor[rgb]{0.31,0.60,0.02}{#1}}
\newcommand{\CommentTok}[1]{\textcolor[rgb]{0.56,0.35,0.01}{\textit{#1}}}
\newcommand{\CommentVarTok}[1]{\textcolor[rgb]{0.56,0.35,0.01}{\textbf{\textit{#1}}}}
\newcommand{\ConstantTok}[1]{\textcolor[rgb]{0.00,0.00,0.00}{#1}}
\newcommand{\ControlFlowTok}[1]{\textcolor[rgb]{0.13,0.29,0.53}{\textbf{#1}}}
\newcommand{\DataTypeTok}[1]{\textcolor[rgb]{0.13,0.29,0.53}{#1}}
\newcommand{\DecValTok}[1]{\textcolor[rgb]{0.00,0.00,0.81}{#1}}
\newcommand{\DocumentationTok}[1]{\textcolor[rgb]{0.56,0.35,0.01}{\textbf{\textit{#1}}}}
\newcommand{\ErrorTok}[1]{\textcolor[rgb]{0.64,0.00,0.00}{\textbf{#1}}}
\newcommand{\ExtensionTok}[1]{#1}
\newcommand{\FloatTok}[1]{\textcolor[rgb]{0.00,0.00,0.81}{#1}}
\newcommand{\FunctionTok}[1]{\textcolor[rgb]{0.00,0.00,0.00}{#1}}
\newcommand{\ImportTok}[1]{#1}
\newcommand{\InformationTok}[1]{\textcolor[rgb]{0.56,0.35,0.01}{\textbf{\textit{#1}}}}
\newcommand{\KeywordTok}[1]{\textcolor[rgb]{0.13,0.29,0.53}{\textbf{#1}}}
\newcommand{\NormalTok}[1]{#1}
\newcommand{\OperatorTok}[1]{\textcolor[rgb]{0.81,0.36,0.00}{\textbf{#1}}}
\newcommand{\OtherTok}[1]{\textcolor[rgb]{0.56,0.35,0.01}{#1}}
\newcommand{\PreprocessorTok}[1]{\textcolor[rgb]{0.56,0.35,0.01}{\textit{#1}}}
\newcommand{\RegionMarkerTok}[1]{#1}
\newcommand{\SpecialCharTok}[1]{\textcolor[rgb]{0.00,0.00,0.00}{#1}}
\newcommand{\SpecialStringTok}[1]{\textcolor[rgb]{0.31,0.60,0.02}{#1}}
\newcommand{\StringTok}[1]{\textcolor[rgb]{0.31,0.60,0.02}{#1}}
\newcommand{\VariableTok}[1]{\textcolor[rgb]{0.00,0.00,0.00}{#1}}
\newcommand{\VerbatimStringTok}[1]{\textcolor[rgb]{0.31,0.60,0.02}{#1}}
\newcommand{\WarningTok}[1]{\textcolor[rgb]{0.56,0.35,0.01}{\textbf{\textit{#1}}}}
\usepackage{graphicx}
\makeatletter
\def\maxwidth{\ifdim\Gin@nat@width>\linewidth\linewidth\else\Gin@nat@width\fi}
\def\maxheight{\ifdim\Gin@nat@height>\textheight\textheight\else\Gin@nat@height\fi}
\makeatother
% Scale images if necessary, so that they will not overflow the page
% margins by default, and it is still possible to overwrite the defaults
% using explicit options in \includegraphics[width, height, ...]{}
\setkeys{Gin}{width=\maxwidth,height=\maxheight,keepaspectratio}
% Set default figure placement to htbp
\makeatletter
\def\fps@figure{htbp}
\makeatother
\setlength{\emergencystretch}{3em} % prevent overfull lines
\providecommand{\tightlist}{%
  \setlength{\itemsep}{0pt}\setlength{\parskip}{0pt}}
\setcounter{secnumdepth}{-\maxdimen} % remove section numbering
\ifLuaTeX
  \usepackage{selnolig}  % disable illegal ligatures
\fi
\IfFileExists{bookmark.sty}{\usepackage{bookmark}}{\usepackage{hyperref}}
\IfFileExists{xurl.sty}{\usepackage{xurl}}{} % add URL line breaks if available
\urlstyle{same} % disable monospaced font for URLs
\hypersetup{
  pdftitle={TPRegresion},
  pdfauthor={Jose Valdes},
  hidelinks,
  pdfcreator={LaTeX via pandoc}}

\title{TPRegresion}
\author{Jose Valdes}
\date{2023-06-05}

\begin{document}
\maketitle

{
\setcounter{tocdepth}{2}
\tableofcontents
}
\begin{Shaded}
\begin{Highlighting}[]
\CommentTok{\#limpio la memoria}
\FunctionTok{rm}\NormalTok{( }\AttributeTok{list=} \FunctionTok{ls}\NormalTok{(}\AttributeTok{all.names=} \ConstantTok{TRUE}\NormalTok{) )  }\CommentTok{\#remove all objects}
\FunctionTok{gc}\NormalTok{( }\AttributeTok{full=} \ConstantTok{TRUE}\NormalTok{ )                 }\CommentTok{\#garbage collection}
\end{Highlighting}
\end{Shaded}

\begin{verbatim}
##          used (Mb) gc trigger (Mb) max used (Mb)
## Ncells 463382 24.8    1001825 53.6   644245 34.5
## Vcells 832758  6.4    8388608 64.0  1635137 12.5
\end{verbatim}

\hypertarget{correlaciuxf3n}{%
\subsection{\texorpdfstring{{1.1.
Correlación}}{1.1. Correlación}}\label{correlaciuxf3n}}

\hypertarget{ejercicio-1.1.-en-el-archivo-grasacerdos.xlsx-se-encuentran-los-datos-del-peso-vivo-pv-en-kg-y-al-espesor-de-grasa-dorsal-egd-en-mm-de-30-lechones-elegidos-al-azar-de-una-poblaciuxf3n-de-porcinos-duroc-jersey-del-oeste-de-la-provincia-de-buenos-aires.-se-pide}{%
\subsubsection{Ejercicio 1.1. En el archivo grasacerdos.xlsx se
encuentran los datos del peso vivo (PV, en Kg) y al espesor de grasa
dorsal (EGD, en mm) de 30 lechones elegidos al azar de una población de
porcinos Duroc Jersey del Oeste de la provincia de Buenos Aires. Se
pide}\label{ejercicio-1.1.-en-el-archivo-grasacerdos.xlsx-se-encuentran-los-datos-del-peso-vivo-pv-en-kg-y-al-espesor-de-grasa-dorsal-egd-en-mm-de-30-lechones-elegidos-al-azar-de-una-poblaciuxf3n-de-porcinos-duroc-jersey-del-oeste-de-la-provincia-de-buenos-aires.-se-pide}}

\hypertarget{a-dibujar-el-diagrama-de-dispersiuxf3n-e-interpretarlo.}{%
\subsubsection{(a) Dibujar el diagrama de dispersión e
interpretarlo.}\label{a-dibujar-el-diagrama-de-dispersiuxf3n-e-interpretarlo.}}

\begin{Shaded}
\begin{Highlighting}[]
\FunctionTok{library}\NormalTok{(readxl)}
\FunctionTok{library}\NormalTok{(ggplot2)}
\FunctionTok{library}\NormalTok{(MVN)}
\FunctionTok{library}\NormalTok{(gridExtra)}

\NormalTok{grasacerdos}\OtherTok{\textless{}{-}}\FunctionTok{read\_excel}\NormalTok{(}\StringTok{"C:/Users/Josvaldes/Documents/Maestria/Austral/1ano/regresionAvanzada/TPRegresion/TPRegresion/grasacerdos.xlsx"}\NormalTok{)}
\FunctionTok{dim}\NormalTok{(grasacerdos)}
\end{Highlighting}
\end{Shaded}

\begin{verbatim}
## [1] 30  3
\end{verbatim}

\begin{Shaded}
\begin{Highlighting}[]
\FunctionTok{head}\NormalTok{(grasacerdos)}
\end{Highlighting}
\end{Shaded}

\begin{verbatim}
## # A tibble: 6 x 3
##     Obs PV    EGD  
##   <dbl> <chr> <chr>
## 1     1 56,81 16,19
## 2     2 70,40 22,00
## 3     3 71,73 19,52
## 4     4 75,10 31,00
## 5     5 79,65 23,58
## 6     6 51,43 16,58
\end{verbatim}

\begin{Shaded}
\begin{Highlighting}[]
\NormalTok{grasacerdos}\SpecialCharTok{$}\NormalTok{PV }\OtherTok{\textless{}{-}} \FunctionTok{as.numeric}\NormalTok{(}\FunctionTok{gsub}\NormalTok{(}\StringTok{","}\NormalTok{, }\StringTok{"."}\NormalTok{, grasacerdos}\SpecialCharTok{$}\NormalTok{PV))}
\NormalTok{grasacerdos}\SpecialCharTok{$}\NormalTok{EGD }\OtherTok{\textless{}{-}} \FunctionTok{as.numeric}\NormalTok{(}\FunctionTok{gsub}\NormalTok{(}\StringTok{","}\NormalTok{, }\StringTok{"."}\NormalTok{, grasacerdos}\SpecialCharTok{$}\NormalTok{EGD))}
\end{Highlighting}
\end{Shaded}

\begin{Shaded}
\begin{Highlighting}[]
\FunctionTok{ggplot}\NormalTok{(grasacerdos, }\FunctionTok{aes}\NormalTok{(PV, EGD)) }\SpecialCharTok{+} 
  \FunctionTok{geom\_point}\NormalTok{() }\SpecialCharTok{+} \FunctionTok{theme\_minimal}\NormalTok{() }\SpecialCharTok{+} \FunctionTok{labs}\NormalTok{(}\AttributeTok{title =} \StringTok{"Diagrama de Dispersión Peso de Cerdos vs Grasa Dorsal"}\NormalTok{)}
\end{Highlighting}
\end{Shaded}

\includegraphics{TP_files/figure-latex/unnamed-chunk-4-1.pdf}

No se observa corelación entre las variables

\hypertarget{b-calcular-el-coeficiente-de-correlaciuxf3n-muestral-y-expluxedquelo.}{%
\subsubsection{(b) Calcular el coeficiente de correlación muestral y
explíquelo.}\label{b-calcular-el-coeficiente-de-correlaciuxf3n-muestral-y-expluxedquelo.}}

\begin{Shaded}
\begin{Highlighting}[]
\NormalTok{biNormTest }\OtherTok{\textless{}{-}} \FunctionTok{mvn}\NormalTok{(grasacerdos, }\AttributeTok{mvnTest =} \StringTok{"hz"}\NormalTok{)}
\FunctionTok{print}\NormalTok{(biNormTest}\SpecialCharTok{$}\NormalTok{multivariateNormality)}
\end{Highlighting}
\end{Shaded}

\begin{verbatim}
##            Test        HZ   p value MVN
## 1 Henze-Zirkler 0.6379234 0.3891766 YES
\end{verbatim}

Por el resultado se puede sostener el supuesto de una distribución
normal bivariada para estas variables. En tal sentido, se procede a
realizar el test de Pearson para determinar la relación de las
variables:

\begin{Shaded}
\begin{Highlighting}[]
\NormalTok{corCoeff }\OtherTok{\textless{}{-}} \FunctionTok{cor}\NormalTok{(grasacerdos}\SpecialCharTok{$}\NormalTok{PV,grasacerdos}\SpecialCharTok{$}\NormalTok{EGD, }\AttributeTok{method =} \StringTok{"pearson"}\NormalTok{)}
\NormalTok{corCoeff}
\end{Highlighting}
\end{Shaded}

\begin{verbatim}
## [1] 0.2543434
\end{verbatim}

La prueba de correlación de Pearson muestra que existe una correlación
positiva débil entre las variables. Esto significa que hay una tendencia
a que los valores de las variables aumenten juntos, pero la relación no
es muy fuerte.

\hypertarget{c-hay-suficiente-evidencia-para-admitir-asociaciuxf3n-entre-el-peso-y-el-espesor-de-grasa-ux3b1-005.-verifique-los-supuestos-para-decidir-el-indicador-que-va-a-utilizar.}{%
\subsubsection{(c) ¿Hay suficiente evidencia para admitir asociación
entre el peso y el espesor de grasa? (α = 0,05). Verifique los supuestos
para decidir el indicador que va a
utilizar.}\label{c-hay-suficiente-evidencia-para-admitir-asociaciuxf3n-entre-el-peso-y-el-espesor-de-grasa-ux3b1-005.-verifique-los-supuestos-para-decidir-el-indicador-que-va-a-utilizar.}}

Para determinar si hay suficiente evidencia para admitir una asociación
entre el peso y el espesor de grasa, es necesario verificar los
supuestos y luego utilizar un indicador apropiado para evaluar la
correlación entre las variables.

A continuación, se describen los supuestos que se deben verificar antes
de seleccionar el indicador:

1 - Supuesto de normalidad: Se debe verificar si las variables peso y
espesor de grasa siguen una distribución normal. Esto se puede hacer
mediante métodos gráficos, como histogramas o gráficos de Q-Q, y pruebas
estadísticas, como el test de normalidad (por ejemplo, el test de
Shapiro-Wilk).

2 - Supuesto de linealidad: Se debe verificar si la relación entre el
peso y el espesor de grasa es lineal. Esto se puede explorar mediante un
diagrama de dispersión o mediante técnicas de análisis exploratorio de
datos.

3 - Supuesto de homogeneidad de varianzas: Se debe verificar si la
varianza del espesor de grasa es constante en diferentes niveles de
peso. Esto se puede evaluar mediante gráficos de dispersión y pruebas
estadísticas, como el test de Levene.

Una vez que se han verificado los supuestos, puedes seleccionar un
indicador apropiado para evaluar la asociación entre el peso y el
espesor de grasa. Dado que estamos analizando una relación entre dos
variables continuas, el coeficiente de correlación de Pearson sería un
indicador adecuado.

Para determinar si hay suficiente evidencia para admitir la asociación
entre el peso y el espesor de grasa, se puede realizar una prueba de
hipótesis utilizando el coeficiente de correlación de Pearson. El
enunciado de las hipótesis sería:

Hipótesis nula (H0): No hay asociación entre el peso y el espesor de
grasa (ρ = 0). Hipótesis alternativa (HA): Hay asociación entre el peso
y el espesor de grasa (ρ ≠ 0).

\begin{Shaded}
\begin{Highlighting}[]
\NormalTok{corTest }\OtherTok{\textless{}{-}} \FunctionTok{cor.test}\NormalTok{(grasacerdos}\SpecialCharTok{$}\NormalTok{PV,grasacerdos}\SpecialCharTok{$}\NormalTok{EGD, }\AttributeTok{method =} \StringTok{"pearson"}\NormalTok{) }
\NormalTok{corTest}
\end{Highlighting}
\end{Shaded}

\begin{verbatim}
## 
##  Pearson's product-moment correlation
## 
## data:  grasacerdos$PV and grasacerdos$EGD
## t = 1.3916, df = 28, p-value = 0.175
## alternative hypothesis: true correlation is not equal to 0
## 95 percent confidence interval:
##  -0.1166112  0.5630217
## sample estimates:
##       cor 
## 0.2543434
\end{verbatim}

El resultado del test de correlación de Pearson como se mostró en el
punto b corresponde a una correlacion positiva baja entre las variables
y un P-valor de 0.1749942 que seria mayor que el nivel de significancia
\(\alpha\) = 0,05 de la prueba, por tal razon, no se puede afirmar la
presencia de una asociación significativa entre las variables.

\hypertarget{ejercicio-1.2.-los-datos-del-cuarteto-de-anscombe-se-encuentran-en-el-archivo-anscombe.xlsx}{%
\subsubsection{Ejercicio 1.2. Los datos del cuarteto de Anscombe se
encuentran en el archivo
anscombe.xlsx}\label{ejercicio-1.2.-los-datos-del-cuarteto-de-anscombe-se-encuentran-en-el-archivo-anscombe.xlsx}}

Se pide explorar los datos de la siguiente manera:

\hypertarget{a-graficar-los-cuatro-pares-de-datos-en-un-diagrama-de-dispersiuxf3n-cada-uno.}{%
\subsection{(a) Graficar los cuatro pares de datos en un diagrama de
dispersión cada
uno.}\label{a-graficar-los-cuatro-pares-de-datos-en-un-diagrama-de-dispersiuxf3n-cada-uno.}}

\begin{Shaded}
\begin{Highlighting}[]
\CommentTok{\# se observa que el archivo esta incompleto anscombe.xlsx (dimensiones 6x8), se busca en internet y se trabaja con Anscombe\textquotesingle{}s Quartet.xlsx (dimensiones 12x8)}
\NormalTok{anscombe}\OtherTok{\textless{}{-}}\FunctionTok{read\_excel}\NormalTok{(}\StringTok{"C:/Users/Josvaldes/Documents/Maestria/Austral/1ano/regresionAvanzada/TPRegresion/TPRegresion/Anscombe\textquotesingle{}s Quartet.xlsx"}\NormalTok{)}
\FunctionTok{dim}\NormalTok{(anscombe)}
\end{Highlighting}
\end{Shaded}

\begin{verbatim}
## [1] 11  8
\end{verbatim}

\begin{Shaded}
\begin{Highlighting}[]
\FunctionTok{head}\NormalTok{(anscombe)}
\end{Highlighting}
\end{Shaded}

\begin{verbatim}
## # A tibble: 6 x 8
##      X1    X2    X3    X4    X5    X6    X7    X8
##   <dbl> <dbl> <dbl> <dbl> <dbl> <dbl> <dbl> <dbl>
## 1    10  8.04    10  9.14    10  7.46     8  6.58
## 2     8  6.95     8  8.14     8  6.77     8  5.76
## 3    13  7.58    13  8.74    13 12.7      8  7.71
## 4     9  8.81     9  8.77     9  7.11     8  8.84
## 5    11  8.33    11  9.26    11  7.81     8  8.47
## 6    14  9.96    14  8.1     14  8.84     8  7.04
\end{verbatim}

\begin{Shaded}
\begin{Highlighting}[]
\NormalTok{dd1}\OtherTok{=}\FunctionTok{ggplot}\NormalTok{(anscombe, }\FunctionTok{aes}\NormalTok{(X1, X2)) }\SpecialCharTok{+} 
  \FunctionTok{geom\_point}\NormalTok{() }\SpecialCharTok{+} \FunctionTok{theme\_minimal}\NormalTok{() }\SpecialCharTok{+} \FunctionTok{labs}\NormalTok{(}\AttributeTok{title =} \StringTok{"Diagrama de Dispersión X1 vs X2"}\NormalTok{)}
\NormalTok{dd1}
\end{Highlighting}
\end{Shaded}

\includegraphics{TP_files/figure-latex/unnamed-chunk-10-1.pdf}

\begin{Shaded}
\begin{Highlighting}[]
\NormalTok{dd2}\OtherTok{=}\FunctionTok{ggplot}\NormalTok{(anscombe, }\FunctionTok{aes}\NormalTok{(X3, X4)) }\SpecialCharTok{+} 
  \FunctionTok{geom\_point}\NormalTok{() }\SpecialCharTok{+} \FunctionTok{theme\_minimal}\NormalTok{() }\SpecialCharTok{+} \FunctionTok{labs}\NormalTok{(}\AttributeTok{title =} \StringTok{"Diagrama de Dispersión X3 vs X4"}\NormalTok{)}
\NormalTok{dd2}
\end{Highlighting}
\end{Shaded}

\includegraphics{TP_files/figure-latex/unnamed-chunk-11-1.pdf}

\begin{Shaded}
\begin{Highlighting}[]
\NormalTok{dd3}\OtherTok{=}\FunctionTok{ggplot}\NormalTok{(anscombe, }\FunctionTok{aes}\NormalTok{(X5, X6)) }\SpecialCharTok{+} 
  \FunctionTok{geom\_point}\NormalTok{() }\SpecialCharTok{+} \FunctionTok{theme\_minimal}\NormalTok{() }\SpecialCharTok{+} \FunctionTok{labs}\NormalTok{(}\AttributeTok{title =} \StringTok{"Diagrama de Dispersión X5 vs X6"}\NormalTok{)}
\NormalTok{dd3}
\end{Highlighting}
\end{Shaded}

\includegraphics{TP_files/figure-latex/unnamed-chunk-12-1.pdf}

\begin{Shaded}
\begin{Highlighting}[]
\NormalTok{dd4}\OtherTok{=}\FunctionTok{ggplot}\NormalTok{(anscombe, }\FunctionTok{aes}\NormalTok{(X7, X8)) }\SpecialCharTok{+} 
  \FunctionTok{geom\_point}\NormalTok{() }\SpecialCharTok{+} \FunctionTok{theme\_minimal}\NormalTok{() }\SpecialCharTok{+} \FunctionTok{labs}\NormalTok{(}\AttributeTok{title =} \StringTok{"Diagrama de Dispersión X7 vs X8"}\NormalTok{)}
\NormalTok{dd4}
\end{Highlighting}
\end{Shaded}

\includegraphics{TP_files/figure-latex/unnamed-chunk-13-1.pdf}

\begin{Shaded}
\begin{Highlighting}[]
\CommentTok{\#resumen}
\FunctionTok{grid.arrange}\NormalTok{(dd1,dd2,dd3,dd4, }\AttributeTok{ncol =} \DecValTok{2}\NormalTok{, }\AttributeTok{nrow =} \DecValTok{2}\NormalTok{)}
\end{Highlighting}
\end{Shaded}

\includegraphics{TP_files/figure-latex/unnamed-chunk-14-1.pdf}

\hypertarget{b-hallar-los-valores-medios-de-las-variables-para-cada-para-de-datos.}{%
\subsection{(b) Hallar los valores medios de las variables para cada
para de
datos.}\label{b-hallar-los-valores-medios-de-las-variables-para-cada-para-de-datos.}}

\end{document}
